\documentclass{article}
\usepackage[UTF8]{ctex}
\usepackage[T1]{fontenc}
\usepackage[utf8]{inputenc}
\usepackage[colorlinks, linkcolor = black]{hyperref}
\usepackage{latexsym}
\usepackage{amsmath}
\usepackage{geometry}
\usepackage{ulem}
\usepackage{xcolor}
\usepackage{listings}
\usepackage{amsthm}
\usepackage{amssymb}
\usepackage{setspace}
\usepackage{dashrule}
\usepackage{tikz}
\usetikzlibrary{positioning}

\def\rightbrace#1{#1)}

\lstdefinestyle{log}
{
    basicstyle = \ttfamily,
    keywordstyle = \bfseries,
    linewidth = \linewidth,
    xleftmargin=.2\textwidth,
    xrightmargin=.2\textwidth,
    numbers = left,
    numberstyle = \rightbrace,
    frame = none,
    emph = CHECKPOINT,
    emphstyle = \textbf,
    mathescape = true,
    escapeinside = ``,
}
\lstdefinestyle{transaction}
{
    basicstyle = \ttfamily,
    keywordstyle = \bfseries,
    linewidth = \linewidth,
    xleftmargin=.1\textwidth,
    xrightmargin=.2\textwidth,
    numbers = none,
    frame = none,
    emph = {Redo, Undo, write, output, Abort, Write, log},
    emphstyle = \textbf,
    mathescape = true,
}

\geometry{left = 2cm, right = 2cm}
\makeatletter
\let\origthelstnumber\lst@PlaceNumber

\renewcommand{\section}{\@startsection{section}{1}{0mm}{-5ex plus -.5ex minus -.2ex}{3ex plus .2ex}
{\normalfont\large\bfseries}}

\newcommand*\Suppressnumber{%
  \lst@AddToHook{OnNewLine}{%
    \let\lst@PlaceNumber\relax%
%    \advance\c@lstnumber-\@ne\relax% Not really necessary
  }%
}

\newcommand\Reactivatenumber[1]{%
  \global\c@lstnumber#1%
  \global\advance\c@lstnumber\m@ne\relax%
  \lst@AddToHook{OnNewLine}{%
  \let\lst@PlaceNumber\origthelstnumber%
  }%
}
\makeatother
\title{Homework 7}
\author{PB17000297 罗晏宸}
\date{May 5 2020}

\begin{document}
\maketitle

\section{什么是事务的 ACID 性质?请给出违背事务 ACID 性质的具体例子,每个性质举一个例子。}

\paragraph{解}
ACID 性质:原子性、一致性、隔离性、永久性。以下是一些违反这些性质的例子

如果事务在执行过程中因除以零等原因失败,但一部分数据已经写入数据库并存储到磁盘上,那么违反了原子性;如果有完整性约束“两位教师的姓名不能相同”在事务执行后不满足,即出现了两位相同姓名的教师,那么违反了一致性;对于一对事务$T_i$和$T_j$,$T_i$执行过程中修改了教师的年龄,同时$T_j$读取到的该位教师的年龄是修改之后的值,那么违反了隔离性;事务正确执行后,系统发生了崩溃,没有任何事务执行的结果留存,那么违反了永久性。

\section{如果一个存储过程 A 的内部调用了另一个存储过程 B,此时 A 和 B 是否都可以使用事务编程并保证事务的 ACID 性质?请解释你的理由}

\paragraph{解}
不可以,如果存储过程 B 使用事务编程并成功完成执行,即已提交,但在存储过程 A 中需要回滚,则不能保证数据库中的数据能够回滚到存储过程执行前,此时违反了原子性。

\section{下面是一个数据库系统开始运行后的日志记录,该数据库系统支持检查点。}
设日志修改记录的格式为 \lstinline[style = log]{<$T_{id}$, Variable, Old value, New value>},请给出对于题中所示\textcircled{1}、\textcircled{2}、\textcircled{3}三种故障情形下,数据库系统恢复的过程以及数据元素 A, B, C, D, E, F 和 G 在执行了恢复过程后的值。
    \newpage
    \begin{lstlisting}[style = log]
<$T_1$, Begin Transaction>
<$T_1$, A, 10, 40>
<$T_2$, Begin Transaction>
<$T_1$, B, 20, 60>
<$T_1$, A, 40, 75>
<$T_2$, C, 30, 50>
<$T_2$, D, 40, 80>
<$T_1$, Commit Transaction>
<$T_3$, Begin Transaction>
<$T_3$, E, 50, 90>`\tikz[remember picture, baseline] \node [anchor=base, blue] (anchor1) {};`
<$T_2$, D, 80, 65>
<$T_2$, C, 50, 75>
<$T_2$, Commit Transaction>`\tikz[remember picture, baseline] \node [anchor=base, blue] (anchor2) {};`
<$T_3$, Commit Transaction>
<CHECKPOINT>
<$T_4$, Begin Transaction>
<$T_4$, F, 60, 120>
<$T_4$, G, 70, 140>`\tikz[remember picture, baseline] \node [anchor=base, blue] (anchor3) {};`
<$T_4$, F, 120, 240>
<$T_4$, Commit Transaction>
\end{lstlisting}
    \begin{tikzpicture}[overlay, remember picture]
        \node (1) [below right = -0.7em and 7em of anchor1] {\textcircled{1}};
        \foreach \i in {2, 3}
            {
                \node (temp\i) [below = -0.6em of anchor\i] {\phantom{\i}};
                \node (\i) at (1 |- temp\i) {\textcircled{\i}};
            }
        \foreach \i in {1, 2, 3}
        \draw[dashed] (\i) +(-1, 0) -- +(-22em, 0);
    \end{tikzpicture}

    \paragraph{解}
    \subparagraph{\textcircled{1}}
    \lstinline[style = transaction]{Redo: $T_1$},\lstinline[style = transaction]{Undo: $T_2$, $T_3$}
    ,操作:
    \begin{lstlisting}[style = transaction]
Undo:       write(E, 50); output(E);
            write(D, 40); output(D);
            write(C, 30); output(C);
Redo:       write(A, 40); output(A);
            write(B, 60); output(B);
            write(A, 75); output(A);
Write log:  <Abort, $T_2$>
Write log:  <Abort, $T_3$>
\end{lstlisting}
    结果为:A: 75, B: 60, C: 30, D: 40, E: 50, F: 未知, G: 未知

    \subparagraph{\textcircled{2}}
    \lstinline[style = transaction]{Redo: $T_1$, $T_2$},\lstinline[style = transaction]{Undo: $T_3$}
,操作:
\begin{lstlisting}[style = transaction]
Undo:       write(E, 50); output(E);
Redo:       write(A, 40); output(A);
            write(B, 60); output(B);
            write(A, 75); output(A);
            write(C, 50); output(C);
            write(D, 80); output(D);
            write(D, 65); output(D);
            write(C, 75); output(C);
Write log:  <Abort, $T_3$>
\end{lstlisting}
结果为:A: 75, B: 60, C: 75, D: 65, E: 50, F: 未知, G: 未知

\subparagraph{\textcircled{3}}
\lstinline[style = transaction]{Undo: $T_4$}
,操作:
\begin{lstlisting}[style = transaction]
Undo:       write(F, 60); output(F);
            write(G, 70); output(G);
Write log:  <Abort, $T_4$>
\end{lstlisting}
结果为:A: 75, B: 60, C: 75, D: 65, E: 90, F: 60, G: 70
\end{document}
