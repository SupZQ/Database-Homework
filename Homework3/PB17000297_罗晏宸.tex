\documentclass{article}
\usepackage[UTF8]{ctex}
\usepackage[T1]{fontenc}
\usepackage[utf8]{inputenc}
\usepackage{latexsym}
\usepackage{amsmath}
\usepackage{geometry}
\usepackage{ulem}

\geometry{left = 2cm, right = 2cm}
\makeatletter
\renewcommand{\section}{\@startsection{section}{1}{0mm}
                                {-5ex plus -.5ex minus -.2ex}
                                {3ex plus .2ex}
                                {\normalfont\large\bfseries}}
\makeatother
\title{Homework 3}
\author{PB17000297 罗晏宸}
\date{March 25 2020}

\begin{document}
\maketitle

\section{给定下面的基本表:
\protect\\ \indent 学生表 \texttt{Student(\uline{sno}, sname, age)}
\protect\\ \indent 课程表 \texttt{Course(\uline{cno}, cname, type, credit)}
\protect\\ \indent 选课表 \texttt{SC(\uline{sno}, \uline{cno}, score, \uline{term})}
\protect\\ 其中:
\protect\\ \indent \texttt{type}是整型,\texttt{0}表示必修课,\texttt{1}表示选修课,\texttt{2}表示通识课,\texttt{3}表示公选课。
\protect\\ \indent \texttt{credit}表示课程学分。
\protect\\ \indent \texttt{term}表示第几学期,取值范围为\texttt{1-8}。
\protect\\ 请用SQL语句回答下面的查询:}
\subparagraph{(1)} 查询选修了必修课但是缺少成绩的学生学号和姓名
\subparagraph{(2)} 查询已选必修课总学分大于16并且所选通识课成绩都大于75分的学生姓名
\subparagraph{(3)} 查询所有课程总成绩排名在前50\%(向上取整)的学生中必修课平均分最高的前10位同学,要求返回这些学生的学号、姓名、必修课平均分以及课程总成绩(不足10位时则全部返回)
\subparagraph{(4)} 查询每门课程的课程名、课程类型、平均成绩和不及格率,要求结果按通识课、必修课、选修课、公选课顺序排列(提示:课程名可能有重名)
\subparagraph{(5)} \texttt{SC}表中重复的\texttt{sno}和\texttt{cno}意味着该学生重修了课程(在不同的学期里),现在我们希望删除学生重复选课的信息,只保留最近一个学期的选课记录以及成绩,请给出相应的SQL语句

\paragraph{解}
\subparagraph{(1)}
\begin{Large}
\begin{equation*}
    \mathcal{G}_{\text{S\#}, \textbf{avg}(\text{QTY}) \rightarrow \text{AVGQTY}}\big(\Pi_{\text{S\#}, \text{QTY}}(\text{S} \Join \text{SPJ})\big)
\end{equation*}
\end{Large}

\subparagraph{(2)}
\begin{Large}
\begin{equation*}
    \mathcal{G}_{\text{J\#}, \text{P\#}, \textbf{sum}(\text{QTY}) \rightarrow \text{QTY}}(\text{SPJ})
\end{equation*}
\end{Large}

\subparagraph{(3)}
\begin{Large}
\begin{equation*}
    \Pi_{\text{S\#}, \text{sname}}\Big(\sigma_{\text{QTY} > 300}\big(\mathcal{G}_{\text{S\#}, \textbf{sum}(\text{QTY}) \rightarrow \text{QTY}}(\text{S} \Join \text{SPJ})\big)\Big)
\end{equation*}
\end{Large}

\subparagraph{(4)}
\begin{Large}
\begin{align*}
    \text{J} & \leftarrow \text{J} \cup \{\text{'J00'},\, \text{'Sam'},\, \text{'Hefei'}\} \\
    \text{SPJ} & \leftarrow \text{SPJ} \cup \Big(\Pi_{\text{J\#}}\big(\sigma_{\text{J\#} = \text{'J00'}}(\text{J})\big) \times \mathcal{G}_{\text{S\#}, \text{P\#}, \textbf{max}(\text{QTY}) \rightarrow \text{QTY}}(\text{SPJ})\Big)
\end{align*}
\end{Large}

\subparagraph{(5)}
\begin{Large}
\begin{equation*}
    \text{S} \leftarrow \bigg(\text{S} - \sigma_{\text{S\#} = \text{'S1'}}(\text{S}) \cup \Big(\Pi_{\text{S\#}, \text{sname}, \text{status}, \text{'合肥'}}\big(\sigma_{\text{S\#} = \text{'S1'}}(\text{S})\big)\Big)\bigg)
\end{equation*}
\end{Large}

\end{document}
