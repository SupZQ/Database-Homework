\documentclass{article}
\usepackage[UTF8]{ctex}
\usepackage[T1]{fontenc}
\usepackage[utf8]{inputenc}
\usepackage[colorlinks, linkcolor = black]{hyperref}
\usepackage{latexsym}
\usepackage{amsmath}
\usepackage{geometry}
\usepackage{ulem}
\usepackage{xcolor}
\usepackage{listings}
\usepackage{amsthm}
\usepackage{amssymb}
\usepackage{setspace}
\usepackage{dashrule}
\usepackage{tikz}
\usetikzlibrary{positioning}

\def\rightbrace#1{#1)}

\lstdefinestyle{log}
{
    basicstyle = \ttfamily,
    keywordstyle = \bfseries,
    linewidth = \linewidth,
    xleftmargin=.2\textwidth,
    xrightmargin=.2\textwidth,
    numbers = left,
    numberstyle = \rightbrace,
    frame = none,
    emph = {LOCK, READ, WAIT, WRITE, UNLOCK, ROLLBACK},
    emphstyle = \textbf,
    mathescape = true,
    escapeinside = ``,
}

\geometry{left = 2cm, right = 2cm}

\makeatletter
\renewcommand{\section}{\@startsection{section}{1}{0mm}{-5ex plus -.5ex minus -.2ex}{3ex plus .2ex}
{\normalfont\large\bfseries}}
\makeatother

\newcommand{\tikzmark}[3][]
  {\tikz[remember picture, baseline, every node/.style={inner sep=0,outer sep=0}]
    \node [anchor = base, #1](#2) {#3};}

\title{Homework 8}
\author{PB17000297 罗晏宸}
\date{May 19 2020}

\begin{document}
\maketitle

\section{证明:如果一个并发调度 S 中的所有事务都遵循 2PL,则该调度必定是可串化调度。}

\paragraph{解}
如果一个并发调度 S 中的所有事务都遵循 2PL,则加锁操作一定在解锁前。若在冲突图中有关于 $A$ 的边$T_1 \rightarrow T_2$,则 $T_2$ 对 $A$ 的锁的申请一定在 $T_1$ 释放之后才进行,之后$T_1$对$A$不再进行读或写。假设冲突图中有环,则存在申请过$A$的锁的事务在另一事务之后再次申请,这是违背 2PL 的。因此调度是可串化的。

\section{在锁的相容性矩阵中,\lstinline[style = log]{<S, U>}(即 $T_1$ 先持有了某对象上的 S 锁,$T_2$ 再申请同一对象上的 U 锁时)是相容的,而\lstinline[style = log]{<U, S>}是不相容的。请解释一下 DBMS 为什么要设计这样不对称的锁相容性规则。}

\paragraph{解}
若\lstinline[style = log]{<S, U>}不相容,则不论 $T_1$ 是否对对象进行修改,都不允许 任何 $T_2$ 申请 U 锁,这会影响到并发性能。若\lstinline[style = log]{<U, S>}是相容的,则若 $T_1$ 意图对对象进行修改,则不能阻止后续的事务申请同一对象上的 S 锁。

\section{采用了两阶段锁协议的事务是否一定不会出现脏读问题?如果是,请解释理由;如果不是,请给出一个反例。}

\paragraph{解}
不是,反例如下:
\begin{table}[h]
    \centering
    \begin{tabular}{rll}
          & $T_1$                   & $T_2$                                       \\
        1 & \lstinline[style = log]!LOCK($A$);!  &                                             \\
        2 & \lstinline[style = log]!READ($A$, $t$);!  & \lstinline[style = log]!LOCK($A$);!                      \\
        3 & \lstinline[style = log]!$t \leftarrow t + 100$;!  & \lstinline[style = log]!WAIT;!                      \\
        4 & \lstinline[style = log]!WRITE($A$, $t$);!  & \lstinline[style = log]!WAIT;!                      \\
        5 & \lstinline[style = log]!UNLOCK($A$);! & \lstinline[style = log]!WAIT;!                     \\
        6 & \lstinline[style = log]!ROLLBACK;! & \tikzmark{Anchor1}{\lstinline[style = log]!READ($A$, $t$);!} \\
    \end{tabular}
\end{table}
\begin{tikzpicture}[overlay, remember picture, node distance = 1.5cm]
    \node (Mark1) [below right = 0.5 em and 2em of Anchor1] {脏读};
    \draw [->,thick] (Mark1) to [in = -60, out = 180] (Anchor1);
\end{tikzpicture}

\section{判断下面的并发调度是否冲突可串?如果是,请给出冲突等价的串行调度事务顺序;如果不是,请解释理由。}
\begin{large}
    $$
        w_3(D)\ r_1(A)\ w_2(A)\ r_4(A)\ r_1(C)\ w_2(B)\ r_3(B)\ r_3(A)\ w_1(D)\ w_3(B)\ r_4(B)\ r_4(C)\ w_4(C)\ w4(B)
    $$
\end{large}
\paragraph{解}
不是冲突可串的,尝试给出调度事务顺序:
\begin{large}
    \begin{align*}
        w_3(D)\ r_1(A)\ w_2(A)\ r_4(A)\ r_1(C)\ w_2(B)\ r_3(B)\ r_3(A)\ \tikzmark[blue]{Anchor2-a}{$w_1(D)$}\ \tikzmark[blue]{Anchor2-b}{$w_3(B)$}\ r_4(B)\ r_4(C)\ w_4(C)\ w4(B) \\
        w_3(D)\ r_1(A)\ w_2(A)\ \tikzmark[blue]{Anchor3-a}{$r_4(A)$}\ \tikzmark[blue]{Anchor3-b}{$r_1(C)$}\ w_2(B)\ r_3(B)\ r_3(A)\ w_3(B)\ w_1(D)\ r_4(B)\ r_4(C)\ w_4(C)\ w4(B) \\
        w_3(D)\ r_1(A)\ w_2(A)\ r_1(C)\ \tikzmark[blue]{Anchor4-a}{$r_4(A)$}\ w_2(B)\ r_3(B)\ r_3(A)\ w_3(B)\ \tikzmark[blue]{Anchor4-b}{$w_1(D)$}\ r_4(B)\ r_4(C)\ w_4(C)\ w4(B) \\
        \tikzmark[red]{Anchor5-a}{$w_3(D)$}\ r_1(A)\ \tikzmark[red]{Anchor5-b}{$w_2(A)$}\ r_1(C)\ w_1(D)\ w_2(B)\ r_3(B)\ r_3(A)\ w_3(B)\ r_4(A)\ r_4(B)\ r_4(C)\ w_4(C)\ w4(B) %\\
        % w_2(A)\ r_1(A)\ \tikzmark[blue]{Anchor6-a}{$w_3(D)$}\ r_1(C)\ w_1(D)\ \tikzmark[blue]{Anchor6-b}{$w_2(B)$}\ r_3(B)\ r_3(A)\ w_3(B)\ r_4(A)\ r_4(B)\ r_4(C)\ w_4(C)\ w4(B) \\
        % w_2(A)\ \tikzmark[blue]{Anchor7-a}{$r_1(A)$}\ \tikzmark[blue]{Anchor7-b}{$w_2(B)$}\ r_1(C)\ w_1(D)\ w_3(D)\ r_3(B)\ r_3(A)\ w_3(B)\ r_4(A)\ r_4(B)\ r_4(C)\ w_4(C)\ w4(B) \\
        % \underbrace{w_2(A)\ w_2(B)}_{\displaystyle T_2}\,\underbrace{r_1(A)\ r_1(C)\ w_1(D)}_{\displaystyle T_1}\,\underbrace{w_3(D)\ r_3(B)\ r_3(A)\ w_3(B)}_{\displaystyle T_3}\,\underbrace{r_4(A)\ r_4(B)\ r_4(C)\ w_4(C)\ w4(B)}_{\displaystyle T_4}
    \end{align*}
\end{large}
交换无法继续进行,并发调度不是冲突可串的。
\begin{tikzpicture}[overlay, remember picture, node distance = 1.5cm]
    \foreach \i in {2, 3}
    \draw [<->, thick, blue!70!black] (Anchor\i-a) to [in = -120, out = -60] (Anchor\i-b);
    \draw [<->, thick, blue!70!black] (Anchor4-a) to [in = -165, out = -15] (Anchor4-b);
    \draw [<->, thick, red!70!black] (Anchor5-a) to [in = -150, out = -30] (Anchor5-b);
    % \draw [<->, thick, blue!70!black] (Anchor6-a) to [in = -160, out = -20] (Anchor6-b);
\end{tikzpicture}

\section{判断下面的并发调度能否由一个使用 2PL 的调度器生成。如果可以,给出一种可能的加锁解锁的顺序;如果不行,请解释理由。}
\begin{large}
    $$
        w_3(x)\ w_4(y)\ w_1(z)\ w_3(a)\ w_3(y)\ w_2(a)\ w_3(x)\ w_3(x)\ w_1(a)\ w_3(y)\ w_2(z)\ w_1(x)
    $$
\end{large}
\newpage
\paragraph{解}
不可以,一个不满足 2PL 的加锁解锁顺序如下,注意到在释放对$a$的锁之后,事务$T_1$进入缩减阶段,不能再申请新锁,而事务最后的$w_1(x)$要求对$x$增加排它锁,因此并发调度不能由一个使用 2PL 的调度器生成。
\begin{large}
    \begin{equation*}
        \tikzmark{Insert0}{}w_3(x)\tikzmark{Insert1}{$\ $}w_4(y)\tikzmark{Insert2}{$\ $}w_1(z)\tikzmark{Insert3}{$\ $}w_3(a)\tikzmark{Insert4}{$\ $}w_3(y)\tikzmark{Insert5}{$\ $}w_2(a)\tikzmark{Insert6}{$\ $}w_3(x)\tikzmark{Insert7}{$\ $}w_1(a)\tikzmark{Insert8}{$\ $}w_3(y)\tikzmark{Insert9}{$\ $}w_2(z)\tikzmark{Insert10}{$\ $}w_1(x)\tikzmark{Insert11}{}
    \end{equation*}
\end{large}
\begin{tikzpicture}[overlay, remember picture]
    \foreach \i/\j in {0/{$L_3(x)$}, 1/{$L_4(y)$}, 2/{$L_1(z)$}, 3/{$L_3(a)$}, 5/{$U_3(a),\,L_2(a)$}, 9/{$U_1(z),\,L_2(z)$}}
    \node (Mark\i) [blue, below = 1em of Insert\i] {\j};
    \foreach \i/\j in {4/{$U_4(y),\,L_3(y)$}, 7/{$U_2(a),\,L_1(a)$}, 10/{$U_3(x),\,L_1(x)$}}
    \node (Mark\i) [blue, below = 3em of Insert\i] {\j};
    \foreach \i/\j in {11/{$U_1(x),\,U_3(y),\,U_2(z),\,U_1(a)$}}
    \node (Mark\i) [blue, below = 5em of Insert\i] {\j};
    \foreach \i in {0, ...,5,7,9,10, 11}
    \draw [blue, ->,thick] (Mark\i) to [in = -90, out = 90] (Insert\i);
\end{tikzpicture}

\end{document}
