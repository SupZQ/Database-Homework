\documentclass{article}
\usepackage[UTF8]{ctex}
\usepackage[T1]{fontenc}
\usepackage[utf8]{inputenc}
\usepackage{latexsym}
\usepackage{amsmath}
\usepackage{geometry}

\geometry{left = 2cm, right = 2cm}
\makeatletter
\renewcommand{\section}{\@startsection{section}{1}{0mm}
                                {-5ex plus -.5ex minus -.2ex}
                                {3ex plus .2ex}
                                {\normalfont\large\bfseries}}
\makeatother
\title{Homework 1}
\author{PB17000297 罗晏宸}
\date{March 3 2020}

\begin{document}
\maketitle

\section{为什么在数据库的 ANSI/SPARC 体系结构中,外模式需要设计成多个?这么做有什么好处?}

\paragraph{解}
外模式是单个用户的视图,是单个用户所看到的局部数据的逻辑结构和特征的描述。 它是用户与数据库系统的数据接口,对于用户而言,外模式就是数据库,因此如果不同用户在应用需求,看待数据的方式,对数据保密的要求等方面存在差异,则其外模式描述就是不同的。即使对模式中同一数据,在外模式中的结构,类型,长度,保密级别等都可以是不同的。这样的设计可以保证数据库安全性,每个用户只能看见和访问所对应的外模式中的数据,数据库中的其余数据是不可见的。

\section{什么是数据库的逻辑数据独立性?请举例说明。}
\paragraph{解}
数据库的逻辑数据独立性指的是当概念模式发生改变时,只要修改外模式/模式映象,可保持外模式不变,从而保持用户应用程序不变,保证了数据与用户程序的逻辑独立性。例如对于此前中国科学技术大学综合教务系统中对于学校开课数据的改动,在概念模式中将不同老师教学的同一课程从具有不同的课程编号改为具有相同的课程编号和不同的课堂号,随后修改映像以保持学生用户选课的外模式以及教务系统选课界面的网页程序不变,实现了逻辑数据独立性。

\section{为什么关系数据模型要求关系必须满足实体完整性?}
\paragraph{解}
实体完整性指关系模式的主码不可为空,即组成主码的所有属性均不可取空值。一个关系对应现实世界中一个实体集。现实世界中的实体是可以相互区分、识别的,也即它们应具有某种唯一性标识。在关系模式中,以主码作为唯一性标识,而组成主码的所有属性不能取空值,否则,表明关系模式中存在着不可标识或不确定的实体,这与现实世界的实际情况相矛盾,这样的实体就不是一个完整实体。

\section{关系数据模型要求外码所引用的属性必须是候选码,我们能否放松要求让外码引用非码属性?试给出你的分析 。}
\paragraph{解}
不能放松这样的要求。外码的作用是保持数据一致性,完整性,控制存储在参照关系中的数据,而参照完整性约束是指参照关系的任一个外码值必须等于被参照关系中所参照的候选码的某个值或者为空,这实际上在关系数据模型中约束了不在不存在的实体间建立关系。如果放松要求,即外码可以任意引用属性,则可能出现在现实世界中没有实体对应的关系,造成数据冗余和不一致。


\section{现实世界中的数据约束是否都可以通过关系数据模型的三类完整性规则来表示?如果是,请解释理由。如果不是,请给出一个反例。}
\paragraph{解}
不是。例如在地理数据库中,对于地势起伏的连续性变化,水体边缘与不同地形相连以及水体底部地形与边缘的连续性变化,这类数据约束难以通过关系数据模型的完整性规则来表示。

\end{document}