\documentclass{article}
\usepackage[UTF8]{ctex}
\usepackage[T1]{fontenc}
\usepackage[utf8]{inputenc}
\usepackage[colorlinks, linkcolor = black]{hyperref}
\usepackage{latexsym}
\usepackage{amsmath}
\usepackage{geometry}
\usepackage{float}
\usepackage{xcolor}
\usepackage{amsthm}
\usepackage{amssymb}
\usepackage{subcaption}
\usepackage{tikz-B+Tree}

\title{Homework 6}
\author{PB17000297 罗晏宸}
\date{April 26 2020}

\begin{document}
\maketitle

\section{考虑下面的 B+ 树(n = 3):}
\begin{figure}[H]
    \centering
    \begin{tikzpicture}[every node/.style = {B+node,draw}, every path/.style = {Pointer}]
        \node at(    0,     0) (A1) {\B+(13|  |  )};
        \node at( -6em,  -6em) (B1) {\B+( 7|  |  )};
        \node at(  6em,  -6em) (B2) {\B+(23|31|43)};
        \node at(-15em, -12em) (C1) {\B+( 2| 3| 5)};
        \node at( -9em, -12em) (C2) {\B+( 7|11|  )};
        \node at( -3em, -12em) (C3) {\B+(13|17|19)};
        \node at(  3em, -12em) (C4) {\B+(23|29|  )};
        \node at(  9em, -12em) (C5) {\B+(31|37|41)};
        \node at( 15em, -12em) (C6) {\B+(43|47|  )};
        \foreach \i in {1, 2}
            {
                \draw (A1.ptr\i) to (B\i);
            }
        \foreach \i in {1, 2}
            {
                \draw (B1.ptr\i) to (C\i);
            }
        \foreach \i [evaluate=\i as \j using int(\i+2)] in {1, ..., 4}
            {
                \draw (B2.ptr\i) to (C\j);
            }
        \foreach \i/\j in {1/1, 1/2, 1/3, 2/1, 2/2, 3/1, 3/2, 3/3, 4/1, 4/2, 5/1, 5/2, 5/3, 6/1, 6/2}
            {
                \draw (C\i.ptr\j) -- +(0, -1);
            }
        \foreach \i [evaluate=\i as \j using int(\i+1)] in {1, ..., 5}
            {
                \draw (C\i.ptr4) to[out = 0, in = 180] (C\j.proc);
            }
    \end{tikzpicture}
    \label{B+}
\end{figure}
\subparagraph{1)} 画出依次插入了 36, 18, 40 后的 B+ 树;
\subparagraph{2)} 画出在 1) 所得的 B+ 树中依次删除 43, 13, 7 之后最终的 B+ 树

\paragraph{解}
\subparagraph{1)}
依次插入 36, 18, 40 后的 B+ 树如图\ref{B+Answer}(\subref{B+36})(\subref{B+18})(\subref{B+40})所示。
\subparagraph{2)}
依次删除 43, 13, 7 后的 B+ 树如图\ref{B+Answer2}(\subref{B+43})(\subref{B+13})(\subref{B+7})所示。
\begin{figure}[H]
    \centering
    \begin{subfigure}{\textwidth}
        \centering
        \scalebox{0.80}{
            \begin{tikzpicture}[every node/.style = {B+node,draw}, every path/.style = {Pointer}]
                \node at(  0  ,   0  ) (A1) {\B+(13|{\color{blue}37}|)};
                \node at(-12em,  -6em) (B1) {\B+(7||)};
                \node at(  0em,  -6em) (B2) {\B+({\color{blue}23}|{\color{blue}31}|)};
                \node at( 12em,  -6em) (B3) {\B+({\color{blue}43}||)};
                \node at(-18em, -12em) (C1) {\B+(2|3|5)};
                \node at(-12em, -12em) (C2) {\B+(7|11|)};
                \node at( -6em, -12em) (C3) {\B+(13|17|19)};
                \node at(  0em, -12em) (C4) {\B+(23|29|)};
                \node at(  6em, -12em) (C5) {\B+({\color{blue}31}|{\color{red}36}|)};
                \node at( 12em, -12em) (C6) {\B+({\color{blue}37}|{\color{blue}41}|)};
                \node at( 18em, -12em) (C7) {\B+(43|47|)};
                \foreach \i in {1, 2, 3}
                    {
                        \draw (A1.ptr\i) to (B\i);
                    }
                \foreach \i in {1, 2}
                    {
                        \draw (B1.ptr\i) to (C\i);
                    }
                \foreach \i [evaluate=\i as \j using int(\i+2)] in {1, ..., 3}
                    {
                        \draw (B2.ptr\i) to (C\j);
                    }
                \foreach \i [evaluate=\i as \j using int(\i+5)] in {1, 2}
                    {
                        \draw (B3.ptr\i) to (C\j);
                    }
                \foreach \i/\j in {1/1, 1/2, 1/3, 2/1, 2/2, 3/1, 3/2, 3/3, 4/1, 4/2, 5/1, 5/2, 6/1, 6/2, 7/1, 7/2}
                    {
                        \draw (C\i.ptr\j) -- +(0, -1);
                    }
                \foreach \i [evaluate=\i as \j using int(\i+1)] in {1, ..., 6}
                    {
                        \draw (C\i.ptr4) to[out = 0, in = 180] (C\j.proc);
                    }
                \draw [blue] (C5.ptr4) to[out = 0, in = 180] (C6.proc);
            \end{tikzpicture}
        }
        \caption{插入了 36 后的 B+ 树}
        \label{B+36}
    \end{subfigure}
    \\[2em]
    \begin{subfigure}{\textwidth}
        \centering
        \scalebox{0.80}{
            \begin{tikzpicture}[every node/.style = {B+node,draw}, every path/.style = {Pointer}]
                \node at(  0  ,   0  ) (A1) {\B+(13|37|)};
                \node at(-12em,  -6em) (B1) {\B+(7||)};
                \node at(  0em,  -6em) (B2) {\B+({\color{red}18}|23|31)};
                \node at( 12em,  -6em) (B3) {\B+(43||)};
                \node at(-21em, -12em) (C1) {\B+(2|3|5)};
                \node at(-15em, -12em) (C2) {\B+(7|11|)};
                \node at( -9em, -12em) (C3) {\B+({\color{blue}13}|{\color{blue}17}|)};
                \node at( -3em, -12em) (C4) {\B+({\color{red}18}|{\color{blue}19}|)};
                \node at(  3em, -12em) (C5) {\B+(23|29|)};
                \node at(  9em, -12em) (C6) {\B+(31|36|)};
                \node at( 15em, -12em) (C7) {\B+(37|41|)};
                \node at( 21em, -12em) (C8) {\B+(43|47|)};
                \foreach \i in {1, 2, 3}
                    {
                        \draw (A1.ptr\i) to (B\i);
                    }
                \foreach \i in {1, 2}
                    {
                        \draw (B1.ptr\i) to (C\i);
                    }
                \foreach \i [evaluate=\i as \j using int(\i+2)] in {1, ..., 4}
                    {
                        \draw (B2.ptr\i) to (C\j);
                    }
                \foreach \i [evaluate=\i as \j using int(\i+6)] in {1, 2}
                    {
                        \draw (B3.ptr\i) to (C\j);
                    }
                \foreach \i/\j in {1/1, 1/2, 1/3, 2/1, 2/2, 3/1, 3/2, 4/1, 4/2, 5/1, 5/2, 6/1, 6/2, 7/1, 7/2, 8/1, 8/2}
                    {
                        \draw (C\i.ptr\j) -- +(0, -1);
                    }
                \foreach \i [evaluate=\i as \j using int(\i+1)] in {1, ..., 6}
                    {
                        \draw (C\i.ptr4) to[out = 0, in = 180] (C\j.proc);
                    }
                \draw [blue] (C3.ptr4) to[out = 0, in = 180] (C4.proc);
            \end{tikzpicture}
        }
        \caption{插入了 36, 18 后的 B+ 树}
        \label{B+18}
    \end{subfigure}
    \\[2em]
    \begin{subfigure}{\textwidth}
        \centering
        \scalebox{0.80}{
            \begin{tikzpicture}[every node/.style = {B+node,draw}, every path/.style = {Pointer}]
                \node at(  0  ,   0  ) (A1) {\B+(13|37|)};
                \node at(-12em,  -6em) (B1) {\B+(7||)};
                \node at(  0em,  -6em) (B2) {\B+(18|23|31)};
                \node at( 12em,  -6em) (B3) {\B+(43||)};
                \node at(-21em, -12em) (C1) {\B+(2|3|5)};
                \node at(-15em, -12em) (C2) {\B+(7|11|)};
                \node at( -9em, -12em) (C3) {\B+(13|17|)};
                \node at( -3em, -12em) (C4) {\B+(18|19|)};
                \node at(  3em, -12em) (C5) {\B+(23|29|)};
                \node at(  9em, -12em) (C6) {\B+(31|36|)};
                \node at( 15em, -12em) (C7) {\B+(37|{\color{red}40}|41)};
                \node at( 21em, -12em) (C8) {\B+(43|47|)};
                \foreach \i in {1, 2, 3}
                    {
                        \draw (A1.ptr\i) to (B\i);
                    }
                \foreach \i in {1, 2}
                    {
                        \draw (B1.ptr\i) to (C\i);
                    }
                \foreach \i [evaluate=\i as \j using int(\i+2)] in {1, ..., 4}
                    {
                        \draw (B2.ptr\i) to (C\j);
                    }
                \foreach \i [evaluate=\i as \j using int(\i+6)] in {1, 2}
                    {
                        \draw (B3.ptr\i) to (C\j);
                    }
                \foreach \i/\j in {1/1, 1/2, 1/3, 2/1, 2/2, 3/1, 3/2, 4/1, 4/2, 5/1, 5/2, 6/1, 6/2, 7/1, 7/2, 7/3, 8/1, 8/2}
                    {
                        \draw (C\i.ptr\j) -- +(0, -1);
                    }
                \foreach \i [evaluate=\i as \j using int(\i+1)] in {1, ..., 6}
                    {
                        \draw (C\i.ptr4) to[out = 0, in = 180] (C\j.proc);
                    }
            \end{tikzpicture}
        }
        \caption{插入了 36, 18, 40 后的 B+ 树}
        \label{B+40}
    \end{subfigure}
    \caption{依次插入了 36, 18, 40 后的 B+ 树}
    \label{B+Answer}
\end{figure}
\begin{figure}[H]
    \centering
    \begin{subfigure}{\textwidth}
        \centering
        \scalebox{0.80}{
            \begin{tikzpicture}[every node/.style = {B+node,draw}, every path/.style = {Pointer}]
                \node at(  0  ,   0  ) (A1) {\B+(13|37|)};
                \node at(-12em,  -6em) (B1) {\B+(7||)};
                \node at(  0em,  -6em) (B2) {\B+(18|23|31)};
                \node at( 12em,  -6em) (B3) {\B+({\color{green!70!black}41}||)};
                \node at(-21em, -12em) (C1) {\B+(2|3|5)};
                \node at(-15em, -12em) (C2) {\B+(7|11|)};
                \node at( -9em, -12em) (C3) {\B+(13|17|)};
                \node at( -3em, -12em) (C4) {\B+(18|19|)};
                \node at(  3em, -12em) (C5) {\B+(23|29|)};
                \node at(  9em, -12em) (C6) {\B+(31|36|)};
                \node at( 15em, -12em) (C7) {\B+(37|40|)};
                \node at( 21em, -12em) (C8) {\B+({\color{green!70!black}41}|47|)};
                \foreach \i in {1, 2, 3}
                    {
                        \draw (A1.ptr\i) to (B\i);
                    }
                \foreach \i in {1, 2}
                    {
                        \draw (B1.ptr\i) to (C\i);
                    }
                \foreach \i [evaluate=\i as \j using int(\i+2)] in {1, ..., 4}
                    {
                        \draw (B2.ptr\i) to (C\j);
                    }
                \foreach \i [evaluate=\i as \j using int(\i+6)] in {1, 2}
                    {
                        \draw (B3.ptr\i) to (C\j);
                    }
                \foreach \i/\j in {1/1, 1/2, 1/3, 2/1, 2/2, 3/1, 3/2, 4/1, 4/2, 5/1, 5/2, 6/1, 6/2, 7/1, 7/2, 8/1, 8/2}
                    {
                        \draw (C\i.ptr\j) -- +(0, -1);
                    }
                \foreach \i [evaluate=\i as \j using int(\i+1)] in {1, ..., 6}
                    {
                        \draw (C\i.ptr4) to[out = 0, in = 180] (C\j.proc);
                    }
            \end{tikzpicture}
        }
        \caption{删除了 43 后的 B+ 树}
        \label{B+43}
    \end{subfigure}
    \\[2em]
    \begin{subfigure}{\textwidth}
        \centering
        \scalebox{0.80}{
            \begin{tikzpicture}[every node/.style = {B+node,draw}, every path/.style = {Pointer}]
                \node at(  0  ,   0  ) (A1) {\B+({\color{green!70!black}17}|37|)};
                \node at(-12em,  -6em) (B1) {\B+(7||)};
                \node at(  0em,  -6em) (B2) {\B+(23|31|)};
                \node at( 12em,  -6em) (B3) {\B+(41||)};
                \node at(-18em, -12em) (C1) {\B+(2|3|5)};
                \node at(-12em, -12em) (C2) {\B+(7|11|)};
                \node at( -6em, -12em) (C3) {\B+({\color{green!70!black}17}|18|19)};
                \node at(  0em, -12em) (C4) {\B+(23|29|)};
                \node at(  6em, -12em) (C5) {\B+(31|36|)};
                \node at( 12em, -12em) (C6) {\B+(37|40|)};
                \node at( 18em, -12em) (C7) {\B+(41|47|)};
                \foreach \i in {1, 2, 3}
                    {
                        \draw (A1.ptr\i) to (B\i);
                    }
                \foreach \i in {1, 2}
                    {
                        \draw (B1.ptr\i) to (C\i);
                    }
                \foreach \i [evaluate=\i as \j using int(\i+2)] in {1, ..., 3}
                    {
                        \draw (B2.ptr\i) to (C\j);
                    }
                \foreach \i [evaluate=\i as \j using int(\i+5)] in {1, 2}
                    {
                        \draw (B3.ptr\i) to (C\j);
                    }
                \foreach \i/\j in {1/1, 1/2, 1/3, 2/1, 2/2, 3/1, 3/2, 3/3, 4/1, 4/2, 5/1, 5/2, 6/1, 6/2, 7/1, 7/2}
                    {
                        \draw (C\i.ptr\j) -- +(0, -1);
                    }
                \foreach \i [evaluate=\i as \j using int(\i+1)] in {1, ..., 6}
                    {
                        \draw (C\i.ptr4) to[out = 0, in = 180] (C\j.proc);
                    }
                \draw [green!70!black] (C2.ptr4) to[out = 0, in = 180] (C3.proc);
            \end{tikzpicture}
        }
        \caption{删除了 43, 13 后的 B+ 树}
        \label{B+13}
    \end{subfigure}
    \\[2em]
    \begin{subfigure}{\textwidth}
        \centering
        \scalebox{0.80}{
            \begin{tikzpicture}[every node/.style = {B+node,draw}, every path/.style = {Pointer}]
                \node at(  0  ,   0  ) (A1) {\B+(17|37|)};
                \node at(-12em,  -6em) (B1) {\B+({\color{green!70!black}5}||)};
                \node at(  0em,  -6em) (B2) {\B+(23|31|)};
                \node at( 12em,  -6em) (B3) {\B+(41||)};
                \node at(-18em, -12em) (C1) {\B+(2|3|)};
                \node at(-12em, -12em) (C2) {\B+({\color{green!70!black}5}|11|)};
                \node at( -6em, -12em) (C3) {\B+(17|18|19)};
                \node at(  0em, -12em) (C4) {\B+(23|29|)};
                \node at(  6em, -12em) (C5) {\B+(31|36|)};
                \node at( 12em, -12em) (C6) {\B+(37|40|)};
                \node at( 18em, -12em) (C7) {\B+(41|47|)};
                \foreach \i in {1, 2, 3}
                    {
                        \draw (A1.ptr\i) to (B\i);
                    }
                \foreach \i in {1, 2}
                    {
                        \draw (B1.ptr\i) to (C\i);
                    }
                \foreach \i [evaluate=\i as \j using int(\i+2)] in {1, ..., 3}
                    {
                        \draw (B2.ptr\i) to (C\j);
                    }
                \foreach \i [evaluate=\i as \j using int(\i+5)] in {1, 2}
                    {
                        \draw (B3.ptr\i) to (C\j);
                    }
                \foreach \i/\j in {1/1, 1/2, 2/1, 2/2, 3/1, 3/2, 3/3, 4/1, 4/2, 5/1, 5/2, 6/1, 6/2, 7/1, 7/2}
                    {
                        \draw (C\i.ptr\j) -- +(0, -1);
                    }
                \foreach \i [evaluate=\i as \j using int(\i+1)] in {1, ..., 6}
                    {
                        \draw (C\i.ptr4) to[out = 0, in = 180] (C\j.proc);
                    }
            \end{tikzpicture}
        }
        \caption{删除了 43, 13, 7 后的 B+ 树}
        \label{B+7}
    \end{subfigure}
    \caption{依次删除了 43, 13, 7 后的 B+ 树}
    \label{B+Answer2}
\end{figure}

\section{假设有如下的键值,现用 5 位二进制序列来表示每个键值的 hash 值。回答问题:}
\begin{table}[H]
    \centering
    \begin{tabular}{ccccccc}
        A [11001] & B [00111] & C [00101] & D [00110] & E [10100] & F [01000] & G [00011] \\
        H [11110] & I [10001] & J [01101] & K [10101] & L [11100] & M [01100] & N [11111]
    \end{tabular}
\end{table}
\subparagraph{1)} 如果将上述键值按 A 到 N 的顺序插入到可扩展散列索引中,若每个桶大小为一个磁盘块,每个磁盘块最多可容纳 3 个键值,且初始时散列索引为空,则全部键值插入完成后该散列索引中共有几个桶?并请写出键值 E 所在的桶中的全部键值。
\subparagraph{2)} 前一问题中,如果换成线性散列索引,其余假设不变,同时假设只有当插入新键值后空间利用率大于 80\%时才增加新的桶,则全部键值按序插入完成后该散列索引中共有几个桶?并请写出键值 B 所在的桶中的全部键值(包括溢出块中的键值)。

\paragraph{解}
\subparagraph{1)} 分别在 B、F、G 插入时加倍桶的数目,最终共有 6(8) 个桶,键值 E 所在的桶中的全部键值如下
\begin{table}[H]
    \centering
    \begin{tabular}{ccc}
        E [10100] & I [10001] & K [10101]
    \end{tabular}
\end{table}
\subparagraph{2)} 分别在 C、E、H、J、M 插入时增加新桶,最终共有 6 个桶,键值 B 所在的桶中的全部键值如下
\begin{table}[H]
    \centering
    \begin{tabular}{ccc}
        B [00111] & G [00011] & N [11111]
    \end{tabular}
\end{table}
\end{document}
